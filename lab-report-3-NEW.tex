% Lab report template, 07/10/2016, v. 2. This template should be used in conjunction with the PDF file generated from it.

% This is the header of your report and will not feature in the report itself. It is useful to write things like the title of the document, the date and what version the document is here. The '%' symbol starts a comment line that will not appear in the final PDF.

% This document was written with TeXstudio and compiled with TeXworks, as TeXstudio cannot compile a pdf itself. Programs like TeXworks or MiKTeX can be used exclusively for writing and compiling a document, but are less user friendly than TeXstudio.

%This template is by no means a guide on how to use LaTeX to write documents, but a brief explanation of the major differences between Microsoft Word and LaTeX is given. LaTeX is an American program, and as such, some of the commands use the American way of spelling i.e. color instead of colour. 

%  USEFUL LINKS FOR LEARNING LaTeX:
%
%   http://en.wikibooks.org/wiki/LaTeX
%
%   http://www.andy-roberts.net/writing/latex/tables
%   http://www.andy-roberts.net/writing/latex/floats_figures_captions
%   http://www.andy-roberts.net/res/writing/latex/symbols.pdf
%

\documentclass[11pt]{article} %This sets the font size and the document class of your report. In this case we use 'article' as that is ideal for shorter reports. 

% LaTeX can be enhanced by the use of packages. These packages can do many things, a few of the most common and useful are used here. They are declared before the document proper, in what is known as the 'preamble'. Packages need to be installed when a .tex file compiles into a .pdf, but should do so automatically.

\usepackage[top=2.54cm, bottom=2.54cm, left=2.75cm, right=2.75cm]{geometry} %This sets the margins of the report.
\usepackage{geometry, tikz, float, pgfplots, xcolor, titlesec, amsmath, url, hanging, siunitx, graphicx, sectsty}
\usepackage{graphicx} % A package allowing insertion of images into the text.

% Choose your citations style by commenting out one of the following groups. If you decide to change style, you should also delete the .bbl file that you will find in the same folder as your .tex and .pdf files.

% IEEE style citation:
\usepackage{cite}         % A package that creates references in the IEEE style. 
\newcommand{\citet}{\cite} % Use with cite only, so that it understands the natbib-specific \citet command
\bibliographystyle{ieeetr} % IEEE referencing (use in conjunction with the cite package)

%% Author-date style citation:
%\usepackage[round]{natbib} % A package that creates references in the author-date style, with round brackets
%\renewcommand{\cite}{\citep} % For use with natbib only: comment out for the cite package.
%\bibliographystyle{plainnat} % Author-date referencing (use in conjunction with the natbib package)


\usepackage{color} % Allows the colour of the font to be changed by using the '\color' command: This is just to support the blue comments in this template...use standard (black) text in your report.

\linespread{1.2} % Sets the spacing between lines of text.
\setlength{\parindent}{0cm}  % Suppresses indentation of text at the start of a paragraph

\begin{document} % This begins the document proper and ends the pre-amble


\begin{titlepage} % Begins the titlepage of the document
\begin{center} % Starts the beginning of an environment where all text is centered.

{\Huge Properties of Gamma Radiation}\\[0.5cm] % [0.5cm] sets the distance between this line and the next.
\textit{Jack Greenberg} and \textit{Jacob Fairham}~\\[0.3cm] % The '\\' starts a new paragraph, and will only work after a paragraph has started, unless we use '~'.
\textit{11017146} and \textit{11074241}~\\[0.3cm]
School of Physics and Astronomy~\\[0.3cm]
University of Manchester~\\[0.3cm]
Second year laboratory report~\\[0.3cm]
November 2023~\\[2cm]


\end{center}
{\Large \textbf{Abstract}}~\\[0.3cm]
 This expirement aimed to analyze the spectra created using gamma ray spectroscopy using thallium-activated sodium iodide scintillation detectors with radioactive isotopes such as $^{22}$Na, $^{60}$Co, and $^{137}$Cs. The peaks of the $^{22}$Na spectrum were then used to calculate the strength of the sodium source and the relative efficiencies of the detector at the different photopeak energies. Finally, different materials were used between the isotopes and the detector to measure the relation between gamma ray detections and the thickness of the barrier between the source and the detector at different energy levels. We found our isotope of $^{22}$Na had a source strength of $11.77\pm1.81$kBq, as compared to its theoretical $15.09$kBq obtained using the initial source strength as well as the half life of the isotope and the amount of time that has passed from the initial source measurement.

\end{titlepage}
\pagenumbering{gobble} % This stops the title page being numbered
\clearpage
\pagenumbering{arabic} % sets the style of page numbering for the report
\setcounter{page}{2} % Starts the numbering at page 2 as typically the first page is not numbered

\newpage % Starts a new page to begin the report on.

\section{Introduction} 
\label{intro}
    Gamma radiation was first discovered and studied in 1900 by Paul Villard (citation) however the first quantitative experiment was carried out by Rurtherford and Andrade in 1914; This was carried out through gamma-ray spectroscopy with a rock salt crystal and proved their wavelengths were shorter than the previously studied X-ray radiation (citation). In 1944, gamma-ray spectroscopy began in earnest with Curran and Baker’s development of the scintillation detector (citation) which, to this day, is still the most popular method for detection of gamma rays. This experiment uses the same detection equipment to explore the three ways gamma radiation interacts with matter: Einstein’s photoelectric effect, Compton scattering and pair production. Each of these interactions produces an electron which can then be analysed to find the original energy of the photon and then processed to create a spectrum of detected photon energies. Exploring properties of gamma radiation is important as it is a powerful tool in imagery and detection. These detection techniques have been utilised to identify the presence of different radioactive isotopes which can aid geological mapping, dating of materials and are also notably used medically with many applications including predominantly imaging of inside human bodies.

\section{Theory}
    \subsection{Decay Spectra}
        The spectra obtained from measuring each source in the open air display four regions of importance. The tall, wide 'photopeaks' correspond to electrons detected through the photoelectric effect, matching the decay patterns of the isotopes. The wide spread before each photopeak represents regions of Compton background radiation, where a photon has hit an electron which is then scattered with energies dependant on the angle of scattering $\theta$. The energy of the electron following compton scattering is given by
        \begin{equation}\label{compton}
            E_{electron} = E_{initial}-\frac{E_{initial}}{1+(\frac{E_{initial}}{m_{electron}c^{2}})(1-cos\theta)}
        \end{equation}
        which comes from (citation).

        equation (1) show that the measured energy has a maximum at $\theta$=$180^{\circ}$ which is where a drop in counts is found known as the 'Compton edge'. Inversely, at the lower energy end of compton scattering the 'backscatter peak' can be found. This is where the scattered photons return to the detector instead of the electrons and then, through the photoelectric effect, cause a low energy electron to be detected.

    \subsection{Source Strength Measurements}
        The decay process for $^{22}$Na involves two $511\unit{\keV}$ photons and one $1275\unit{\keV}$ photon. The detector comprises some fraction of the solid angle $\Omega$. The probability of a detection of a $511\unit{\keV}$ and a $1275\unit{\keV}$ photon is
        \begin{equation}
            P(511\unit{\keV}) = 2\Omega\cdot\epsilon(E=511\unit{\keV})
        \end{equation}
        \begin{equation}
            P(1275\unit{\keV}) = \Omega\cdot\epsilon(E=1275\unit{\keV})
        \end{equation}
        and the probability of the sum peak is the combination of the two probabilities.
        
        The rate of detecton for each of the peaks ($R_E$) is a measurable value and is related to the probability of the peak by
        \begin{equation}
            R_E = S\cdot P(E)
        \end{equation}
        equation (4) along with equations (2) and (3) allows the calculation of the ratio of the efficiencies of the detector at different energies
        \begin{equation}
            \frac{\epsilon(E=511\unit{\keV})}{\epsilon(E=1275\unit{\keV})} = \frac{R_{511}}{2R_{1275}}
        \end{equation}
        and the calculation of the source strength by
        \begin{equation}
            S = \frac{R_{511}\cdot R_{1275}}{R_{1786}}
        \end{equation}
    \subsection{Gamma Ray Absorption}
        It can be shown through a simple differential calculation that
        \begin{equation}
            I = I_0 e^{\frac{x}{\tau}}
        \end{equation}
        where $\tau$ is a function of the energy/wavelength of the photon. Therefore, by categorizing the data by photopeak energy and plotting the rate of detection by the thickness of the material for each separate material, we can calculate different values of $\tau$ and relate the absorption of the materials to each other.

% \citep gives (Morin 2008), whereas \citet gives Morin (2008) which you should use if the citation is grammatically part of the sentence


% References are called to a particular place in the text by the '\cite' command. Each individual reference has an idividual label that refers to it, and it is this that is used to call it. More information on this will be given in the references section.

\section{Experimental approach}
    This experiment uses a thallium-activated sodium iodide scintillation detector to measure the spectrum of photon energies released. This consists of a rock-salt crystal connected to a photomultiplier tube (PMT) when connected to a multi-channel analyser which converts it to data the computer can process. 
    The radiation enters the NaI crystal and excites the electrons. The thallium doping of the crystal then ensures the photon released in the ensuing deexcitation is in the visible light spectrum. This is then incident with the photocathode where an electron is liberated through the photoelectric effect. The PMT then amplifies this signal by leading the electron through cups charged with a large potential difference which cause acceleration and therefore the dislodging of more electrons. By the time the signal reaches the anode at the end, the signal has been multiplied ~1,000,000x.
    This electrical signal is then sent to the multi-channel analyser which processes it into different channels corresponding to energy levels of photons. To give specific energies, data must be recorded for each source and then used with the known values of photopeak energies (citation) to calibrate each channel to an energy value.

    \begin{figure}[H] \centering \label{calibration}
        \includegraphics[scale=0.5]{assets/Calibration.png}
        \caption{LSFR plot for conversion from channels to energy}
    \end{figure}

    Figure (1) shows the relation between channel and energy to be
    \begin{equation}\label{C_to_E}
        E = 1.25C + 1.66
    \end{equation}
    where E is an energy in $\unit{\keV}$ and C is the channel number.

    the energy values obtained from running the channels through equation (2) can then be inputted to the software 'Maestro' which automatically converts all channels to their corresponding energies. In principle the channel would have some error related to the full-width half-maximum of the gaussian fit made around the peak, but such an error is not relevant to the results of this experiment. The purpose of this fit is merely to establish the relation between channels and energy values.


\section{Results}
    \subsection{Decay Spectra}
        \begin{figure}[H] \centering \label{na22}
            \includegraphics[scale=0.3]{assets/na22_log_annotated.png}
            \caption{Spectrum of Sodium 22 - Log scale counts by energy}
        \end{figure}
        The Sodium-22 spectrum also displays a 'sum peak' where the detector registers both products
        \begin{figure}[H] \centering \label{co60}
            \includegraphics[scale=0.3]{assets/co60_log_annotated.png}
            \caption{Spectrum of Cobalt 60 - Log scale counts by energy}
        \end{figure}
        \begin{figure}[H] \centering \label{cs137}
            \includegraphics[scale=0.3]{assets/cs137_log_annotated.png}
            \caption{Spectrum of Cesium 137 - Log scale counts by energy}
        \end{figure}
    \subsection{Source Strength Measurements}
    \subsection{Gamma Ray Absorption}
        \begin{figure}[H] \centering \label{exponentials}
            \includegraphics[scale=0.4]{assets/NA511.png}
            \includegraphics[scale=0.4]{assets/NA1275.png}
            \includegraphics[scale=0.4]{assets/CO1173.png}
            \includegraphics[scale=0.4]{assets/CO1332.png}
            \includegraphics[scale=0.4]{assets/CS662.png}
            \caption{Counts per second measured for different thicknesses of different materials - each graph is of a different photopeak. Fit data shown below}
        \end{figure}
        \begin{table} % Begins the table environment
            \begin{center}
            \caption{Height of the meniscus for different magnetic fields.}
            \label{tab:example}
			\begin{tabular}{ ||c|c|c|c|c|c|| }
				$\tau$ & NA511 & NA1275 & CS662 & CO1173 & CO1332\\ 
				Aluminium & $0.027\pm0.002$ & $0.042\pm0.005$ & $0.022\pm0.003$ & $0.062\pm0.043$ & $0.068\pm0.047$\\
				Iron & $0.015\pm0.007$ & $0.022\pm0.015$ & $0.013\pm0.002$ & $0.055\pm0.125$ & $0.031\pm0.039$\\
				Lead & $0.014\pm0.002$ & $0.026\pm0.009$ & $0.008\pm0.001$ & $0.031\pm0.021$ & $0.018\pm0.006$\\
				$\chi^2$ & & & & & \\ 
				Aluminium & $13.66$ & $4.47$ & $0.39$ & $0.59$ & $3.03$\\
				Iron & $0.14$ & $0.49$ & $0.50$ & $4.14$ & $0.65$\\
				Lead & $3.24$ & $0.80$ & $0.24$ & $0.23$ & $2.34$\\
			\end{tabular}
            \smallskip % If you want to put any text under the table, e.g. more detailed column descriptions, leave a space, or else the text will start to the right of the table.
                Table 1: Values of $\tau$ and $\chi^2$ for each photopeak broken down by material
            \smallskip % inserts a small vertical space
            \end{center}
        \end{table}

% For more information on using tables, figures and cross-references in LaTeX, consult Andy Roberts' website (links at top of this document).


\section{Conclusions}
    conclusion

\bibliography{report_template_library} % Specifies the bibliography file where our references are stored. If the library file and document are not in the same folder then the file path must also be included.

{\color{blue} [This section does not contain any written text, just the references themselves.  LaTeX creates it
automatically using a reference library file ({\tt report\_template\_library.bib} in this case), from which it picks entries which have
been cited in the text using the \textbackslash cite\{label reference\} command. See the .bib file for information on how to enter references. You can use one of two reference styles: either author-date, where the citation in the text reads, e.g. (Morin 2008), 
and the references are listed alphabetically by author, or numerical (“IEEE”) referencing, where the citation reads [1] and the references are in numerical order by first use in the text. Do not include the lab script as a reference.]\\

 [Once you progress to the later years of your degree, specifically your MPhys project, you may find that a dedicated reference manager such as Mendelay or EndNote is more helpful to manage the large number of references you will have. These programs can also generate .bib files automatically for your reports.]}\\

{\color{blue}{\Large [Optional sections]}\\

 [There are a number of addition sections that can sometimes be required in a report. Typically these are not required, but can on occasion be helpful. Typically this will be acknowledgements or appendices. Material in appendices is by definition not essential to your report and there is hardly ever a good reason to include it. Remember that your marker has to read a lot of reports and will appreciate brevity!]\\

{\Large [Additional information]}\\

 [Please note that this template is not meant to be a complete guide to writing a lab report.
Basic rules are given on the Blackboard site for 1st-year lab (PHYS10180). The first-year mark sheet serves as a comprehensive summary of basic formatting rules. Marking criteria in second year give much higher weight to complete and accurate description of the experiment.]\\

{\Large [A note on writing style]}\\

 [The style of a lab report should approach that of a scientific paper. This is characterised by clarity and conciseness without verbosity or repetition. The past tense should be used to describe what has been done, and the present tense should be used to describe ongoing situations such as theory. Tenses should not be mixed within a single section. Both the passive voice (e.g. measurements were made) and the active voice (e.g. we concluded) are suitable, but the report should focus on the science and not the individual, so use of personal pronouns such as `I' and  `we' should not become obtrusive. Conditional language should be avoided wherever possible.]\\

{\Large [Writing values]}\\

[Whenever a value is given it should always have units and associated uncertainty, rounded sensibly and with both the values an 
uncertainties quoted to the same sensible level of precision. Units should be written in roman font, and italic font should be used for 
symbols, e.g. \mbox{$m = 15.2 \pm 0.1$~kg}. Use a non-breaking space to make sure values and units do not spread across line 
breaks.]\\
% ~ is the non-breaking space. 
% In this case the line-break wanted to go after the equals sign, which is suppressed by putting the equation in \mbox{}

 [Document produced by Aniela Rodak - November 2014\\
 Last edited by Paddy Leahy - October 2016; e-mail: j.p.leahy@manchester.ac.uk]
}

\end{document}
