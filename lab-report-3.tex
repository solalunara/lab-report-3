%! TeX program = xelatex
\documentclass[12pt, a4paper]{article}
\usepackage{geometry, tikz, float, pgfplots, xcolor, titlesec, amsmath, url, hanging, siunitx, graphicx, sectsty}
\usepackage[utf8]{inputenc}
\usepackage[skip=3pt]{parskip}
\usepackage[none]{hyphenat}
\usepackage[no-math]{fontspec}   % only changes normal font
% Tells LaTeX the images are kept in the "images" folder under the main directory
\graphicspath{ {./assets/} }
\pgfplotsset{compat=1.18, width=10cm}
% Spacing of sections: 0pt on the left, 18pt above, and 12pt below
\titlespacing\section{0pt}{18pt}{12pt}
% font size 12 for the sections
\sectionfont{\fontsize{12}{15}\selectfont}

% Setting font
\setromanfont{Arial}

% Setting strict margins
\sloppy

\begin{document}

\begin{center}

% Extra space at the top of the document
\noindent \\[10pt]
\color{black}

\thispagestyle{empty} % No page number for this page

\Large{\textbf{Properties of Gamma Radiation}} \\[30pt]
\normalsize \textit{Jack Greenberg, Jacob Fairham}\\[5pt]
\textit{11017146,  11074241}\\[20pt]
Department of Physics and Astronomy \\[5pt]
The University of Manchester \\[20pt]
First Year Laboratory Report 2 \\[20pt]
January 2023 \\[25pt]

\end{center}

\textbf{Abstract}\\[12pt]
This expirement aimed to analyze the spectra created using gamma ray spectroscopy using thallium-activated sodium iodide scintillation detectors with radioactive isotopes such as $^{22}$Na, $^{60}$Co, and $^{137}$Cs. The peaks of the $^{22}$Na spectrum were then used to calculate the strength of the sodium source and the relative efficiencies of the detector at the different photopeak energies. Finally, different materials were used between the isotopes and the detector to measure the relation between gamma ray detections and the thickness of the barrier between the source and the detector at different energy levels. We found our isotope of $^{22}$Na had a source strength of $11.77\pm1.81$kBq, as compared to its theoretical $15.09$kBq obtained using the initial source strength as well as the half life of the isotope and the amount of time that has passed from the initial source measurement.

\noindent 

\pagebreak

\section{Introduction}
	<insert introduction here>

\section{Experimental Setup}
	this is how you would include a graph!\\\\
	\textbackslash begin\{figure\}[H] \textbackslash centering\\
	\indent	\textbackslash includegraphics\{assets/circdiag.png\}\\
	\indent	\textbackslash caption\{Circut diagram\}\\
	\textbackslash end{figure}\\\\
	you would put more text here, after the graph

\section{Theory}
	here's how you insert an equation!\\\\
	\textbackslash begin\{equation\}\\
	\indent <your equation here>\\
	\textbackslash end\{equation\}

\section{Procedure}
	more proceedure things

\section{Uncertainty Calculations}
	you know the drill

\section{Data}
	Raw data is availible electronically[2].

\section{Analysis}
	analysis

\section{Conclusion}
	conclusion

\section*{References}

	\begin{hangparas}{.25in}{1}
		[1] Ocaya, Richard. (2006). An experiment to profile the voltage, current and temperature behaviour of a P-N diode. European Journal of Physics. 27. 625. 10.1088/0143-0807/27/3/015.
		
		[2] Daussy, C. et al. "Direct Determination of the Boltzmann Constant by an Optical Method". Phys. Rev. Lett. 98. (2007): 250801.
	\end{hangparas}

\end{document}
